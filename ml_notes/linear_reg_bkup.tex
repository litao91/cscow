\section{Linear Discriminant functions}
A discriminant is a function that take an input vector $\bx$ and assigns
it to one of $K$ classes:
\[g_i(\bx|\bw_i, w_{i0}) =\bw_i^T\bx + w_{i0}\] generalized use base
function $g_i(\bx) = \sum_{j=1}^k w_j \phi_{ij}(\bx)$
\subsection{Two classes}
\[g(\bx) = g_1(\bx) - g_2(\bx) \] 
Choose $C_1$ if $g(\bx)>0$

\subsection{Geometry Interpretation}
The decision boundary is defined by 
\[ g(\bx) = 0\]
Corresponds to a $(D-1)$ dimensional hyperplane within the $D$-dimensional
space.

    Express any point as 
    \[\bx = \bx_p  + r\frac{\bw}{\|\bw\|}\]
    where $x_p$ is the  projection of $\bx$ onto hyperplane. $r$ distance
    from $\bx$ to hyper plane.plane, we have $r=\frac{g(\bx)}{\|\bw\|}$
\subsection{Multi classes} 
K discriminant:
\[g_i(\bx |\bw_i, w_{i0}) = \bw_i^T + w_{i0}\]
Linear separable: \[g_i(\bx|\bw_i, \bw_{i0}) > 0\] if $\bx \in
        C_i$
        
        Choose $C_i$ if \[g_i(\bx) = \max_{j=1}^K g_i(\bx)\]

\subsection{Pairwise Separation} 
    Discriminant function for class $i$ and $j$:
        \[
            g_{ij} (\x|\bw_{ij}, w_{ij0}) = \bw_{ij}^T\x + w_{ij0} =
            \begin{cases}
                > 0 & \mbox{if } \x \in C_i\\
                \leq 0 & \mbox{if } \x \in C_j\\
                \mbox{don't care} & \mbox{if } \x \in C_k, k\neq i, k\neq j
            \end{cases} \]
\section{Logistic Discrimination}
\subsection{Two classes}
     Assume that the log likelihood ratio is linear:
        \[\log\frac{p(\x|C_1)}{p(\x|C_2)} = \bw^T\x + w_0^o \]

    Using Baye's rule we have;
        \begin{align*}
            logit(P(C_1|x)) & = \log\frac{p(C_1|\x)}{1 - p(C_1|\x)} \\
            & = \log\frac{p(\x|C_1)}{p(\x|C_2)} + \log\frac{p(C_1)}{p(C_2)} \\
            & = \bw^T\x + w_0
        \end{align*}
        where $w_0 = w_0^o + \log\frac{P(C_1)}{P(C_2)}$

       Rearranging terms: 
        \begin{align*} y  & = sigmoid(\bw^T\x + w_0) \\
            & =  \hat{P}(C_1|\x) = \frac{1}{1 + \exp\left[-(\bw^T\x + w_0 )\right]}
        \end{align*}
        As our estimator of $P(C_1|\x)$

        \subsubsection{Gradient Decent}
    In the discriminant-based approach, the parameters are those of the
        discriminants, and they are \emph{optimized to minimize the
        classification error}
        \begin{description}
    \item[Error]$\bw$ denotes the set of parameters and $E(\bw|\cX)$ is the
        error parameters $\bw$ on the given training set 
        $\cX$, we look for: 
        \[ \bw*= \arg\min_{\bw}E(\bw|\cX)\]
        No analytical solution
    \item[Gradient Vector] When $E(\bw)$ is a differentiable function of a
        vector of variables, we have the gradient vector composed of the
           partial derivatives:
            \[ \nabla_\bw E = {\left[ \frac{\partial E}{\partial w_1},
                \frac{\partial E}{\partial w_2}, \dots, \frac{\partial
                E}{\partial w_d} \right]}^T
                \]
            \item [Gradient Descent] starts from a \emph{random} $\bw$, at each
                step, update $w$ in a \emph{opposite direction} of the gradient:
                \[\Delta w_i = - \eta\frac{\partial E}{\partial w_i}, \forall i
                    \]
                \[ w_i = w_i + \Delta w_i \]
                $\eta$ is \emph{step size}, or \emph{learning factor}

                When we get to minimum, the derivative is 0 and the procedure
                terminates.

            \item This indicates that the procedure finds the nearest minimum
                that can be \emph{local minimum}. There is no guarantee of
                finds the nearest minimum that can be a local minimum
    \item [Learning parameters]
        Given a sample of two classes, $\mathcal{X} = {\x^{(l)}, \r^{(l)}}$, where $\r^{(l)}
        = 1$ if $\x\in C_1$

        We assume $\r^{(l)}$, given $\x^{(l)}$ is Bernoulli with probability $y^{(l)} =
        p(C_1|\x^{(l)})$:
        \[\r^{(l)}|\x^{(l)} \sim Bernoulli(y^{(l)}) \]
        Note that in this discriminant-based approach, we model directly $\r|\x$
        The sample likelihood is:
        \[ L(\bw,w_0|\cX) = \prod_t{(y^{(l)})}^{(r^{(l)})}{(1-y^{(l)})}^{1-r^{(l)}}
            \]
        We can always turn it in an error function to minimize: $E = -\log L$
        So we have \emph{cross-entropy}:
        \[ E(\bw, w_0| \cX) = -\sum_t \r^{(l)}\log y^{(l)} + (1-r^{(l)})\log(1 - y^{(l)}) \]

        We use gradient descent to minimize cross-entropy 
        If $y = sigmoid(a) = \frac{1}{1 + \exp(-a)}$, its derivative is given as:
        \[ \frac{dy}{da} = y(1-y) \]
        and we get the following update equations:
        \begin{align*}
            \Delta w_j & = -\eta \frac{\partial E}{\partial w_j} = \eta \sum_t
            ({\frac{r^{(l)}}{y^{(l)}}} - \frac{1-r^{(l)}}{1-y^{(l)}}x^{(l)}_j \\
            & = \eta \sum(r^{(l)} - y^{(l)}) x^{(l)}_j\mbox{, }= 1,\dots, d 
        \end{align*}
        \[
            \Delta w_0 = -\eta \frac{\partial E}{\partial w_0} = \eta\sum(r^{(l)} -
            y^{(l)}) 
        \]
\end{description}
